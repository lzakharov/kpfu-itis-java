\documentclass{beamer}

%%% Проверка используемого TeX-движка %%%
\usepackage{ifxetex}

%%% Кодирование и шрифты %%%
\ifxetex
\usepackage{polyglossia}                         % Поддержка многоязычности
\usepackage{fontspec}                            % TrueType-шрифты
\else
\usepackage{cmap}                                % Улучшенный поиск русских слов в полученном pdf-файле
\usepackage[T2A]{fontenc}                        % Поддержка русских букв
\usepackage[utf8]{inputenc}                      % Кодировка utf8
\usepackage[english, russian]{babel}             % Языки: русский, английский
\fi
\usepackage{listings}

%%% Математические пакеты %%%
\usepackage{amsthm,amsfonts,amsmath,amssymb,amscd} % Математические дополнения от AMS

%%% Оформление абзацев %%%
\usepackage{indentfirst}                           % Красная строка

%%% Оформление кода %%%
\lstset{
    language=Java,
    basicstyle=\tiny\sffamily,
    numbers=left,
    numberstyle=\tiny,
    tabsize=4,
    columns=fixed,
    showstringspaces=false,
    showtabs=false,
    keepspaces,
    commentstyle=\color{red},
    keywordstyle=\color{blue}
}

% Стиль презентации
\usetheme{Warsaw}

\newcommand{\br}{\pause \linebreak \linebreak}
\newcommand{\nl}{\pause \linebreak}

\begin{document}
\title{Сортировка расчёской}
\author{Лев Захаров}
\institute{Высшая школа ИТИС КФУ}
\date{Казань, 2015}
% Создание заглавной страницы
\frame{\titlepage}
% Автоматическая генерация содержания
\frame{\frametitle{Содержание}\tableofcontents}

\section{Вступление}
\begin{frame}
    \frametitle{Вступление}
        Сортировка пузырьком, пожалуй, самая простая и известная сортировка. Алгоритм является простым для понимания и легко реализуемым.
        Однако данная сортировка эффективна лишь для небольших массивов. Тем не менее, существуют модификации, позволяющие ускорить работу алгоритма.
        Одной из таких модификаций является \alert{сортировка расческой} или \alert{comb sort}.
\end{frame}

\section{Алгоритм}
\begin{frame}
    \frametitle{Во всем виноваты черепашки}
        В 1980 году Влодзимеж Добосиевич пояснил почему пузырьковая и производные от неё сортировки работают так медленно. \textit{Это всё из-за черепашек.}
        \br
        \alert{Черепаха} $-$ элемент с относительно маленьким значением, находящийся в конце списка.
        \nl
        \alert{Кролик} $-$ элемент с относительно большим значением, находящийся в начале списка.
        \br
        В процессе сортировки черепашки сдвигаются только на одну позицию за один проход. С другой стороны \textit{кролики} двигаются достаточно быстро.

\end{frame}

\begin{frame}
    \frametitle{Сортировка расческой}
    Основная идея сортировки расческой заключается в том, чтобы первоначально выбирать расстояние между сравниваимыми элементами больше еденицы.
    \br
    Для первого прохода шаг равен частному от деления размера массива на \alert{фактор уменьшения}.
    \nl
    На каждом последующем шаге будем делить расстояние между сравниваемыми элементами на \alert{фактор уменьшения}.
    \br
    Так продолжается до тех пор, пока разность индексов сравниваемых элементов не достигнет единицы.
    Дальше массив досортировывается пузырьком.
\end{frame}

\begin{frame}
    \frametitle{Фактор уменьшения}
    Опытным и теоретическим путем было установлено оптимально значение \alert{фактора уменьшения}:
    \begin{center}
        \Large $\frac{1}{1 - \frac1{e^{\varphi}}} \thickapprox 1.247330950103979$
    \end{center}
    , где $\varphi$ есть золотое сечение, т.е. $\varphi = \frac{1 + \sqrt{5}}2$.
\end{frame}

\section{Реализация}
\begin{frame}
    \frametitle{Реализация}
    \lstinputlisting{CombSort.java}
\end{frame}

\section{Оценка сложности}
\begin{frame}
    \frametitle{Оценка сложности}
\end{frame}

\section{Вывод}
\begin{frame}
    \frametitle{Вывод}
\end{frame}

\end{document}